\documentclass[11pt]{article}

\usepackage{fullpage}

\begin{document}

\title{Notes: Prolog, ILP, ALP, TAL, ASP and examples}

\author{Vitalii Protsenko}

\date{\today}         % inserts today's date

\maketitle

\section{Prolog}

\paragraph{}

Based on rules, ie \texttt{dog(a).} means the fact \texttt{'a is a dog'}.
Rules can be derived from other rules by implications \texttt{ :- } or, more naturally,  $\leftarrow$. For example we can write \texttt{animal(A) :- dog(A).}, which would mean \texttt{A is an animal $\leftarrow$ A is a dog}. Worth mentioning that A is a variable, meaning that $\forall A$ : $A$ is a dog $\rightarrow$ $A$ is an animal.


\end{document}